

\usetikzlibrary
{
  calc,         %用以计算坐标
  shadings,     %用以添加阴影或者渐变
}

%=============[定义参数]==============
% - - - - - - [定义脸部轮廓参数]- - - - - - -
\newcommand{\RadiusOutline}{{sqrt(5)}}
\newcommand{\ColorFace}{yellow!45!orange!70!white}

% - - - - - - [定义嘴巴轮廓参数]- - - - - - -
\newcommand{\RadiusMouth}{{sqrt(3)}}

% - - - - - - [定义眼眶参数]- - - - - - -
\newcommand{\CoordinateOrbitLeft}{(-1,-1)}
\newcommand{\CoordinateOrbitRight}{( 1,-1)}
\newcommand{\RadiusOrbitOut}{{sqrt(5)}}
\newcommand{\RadiusOrbitIn}{2}
\newcommand{\AngleOrbit}{180*3/8}
\newcommand{\RadiusOrbitClose}
           {(\RadiusOrbitOut-\RadiusOrbitIn)/2}
%
\def\OrbitPath#1#2
% #1 起始点
% #2 选项, 可选\draw, \clip.
{
  \path [#2] (#1)
        arc [start angle = 90, end angle = 180 - \AngleOrbit,
             radius = \RadiusOrbitOut]
        arc [start angle = 180 -\AngleOrbit, end angle = 360-\AngleOrbit,
             radius = \RadiusOrbitClose]
        arc [start angle = 180 -\AngleOrbit, end angle = \AngleOrbit,
             radius = \RadiusOrbitIn]
        arc [start angle = \AngleOrbit - 180, end angle = \AngleOrbit,
             radius = \RadiusOrbitClose]
        arc [start angle = \AngleOrbit, end angle = 90,
             radius = \RadiusOrbitOut]
        -- cycle;
}
\def\OrbitDraw#1{\OrbitPath{#1}{draw}}
\def\OrbitClip#1{\OrbitPath{#1}{clip}}

% - - - - - - [定义瞳仁参数]- - - - - - -
\newcommand{\RadiusPositionPupil}{(\RadiusOrbitOut+\RadiusOrbitIn)/2}
\newcommand{\AnglePupil}{110}
\newcommand{\RadiusPupil}{\RadiusOrbitClose}
\newcommand{\CoordinatePupilLeft}
           {($(NodeOrbitLeft) + (\AnglePupil:{\RadiusPositionPupil})$)}
\newcommand{\CoordinatePupilRight}
           {($(NodeOrbitRight) + (\AnglePupil:{\RadiusPositionPupil})$)}
%
% - - - - - - [定义眉毛参数]- - - - - - -
\newcommand{\XStartBrow}{0.5}
\newcommand{\YStartBrow}{1.4}

\newcommand{\XTopBrow}  {0.9}
\newcommand{\YTopBrow}  {2.2}

\newcommand{\XEndpBrow} {1.2}
\newcommand{\YEndBrow}  {1.7}

\newcommand{\ThickBrow} {0.1}

% - - - - - - [定义红晕参数]- - - - - - -
\newcommand{\XBlushCentre}{1.4}
\newcommand{\YBlushCentre}{0.6}
\newcommand{\CoordinateRightBlushCcentre}{(\XBlushCentre,\YBlushCentre)}
\newcommand{\CoordinateLeftBlushCcentre}{(-\XBlushCentre,\YBlushCentre)}

\newcommand{\XRadBlush}{0.4}
\newcommand{\YRadBlush}{0.2}



\newcommand{\funny}[1]
% #1 用以给定坐标变换
{
  \begin{scope}[#1,scale = 0.15]
    %=============[绘制脸蛋]==============
    \begin{scope}
      [very thin, draw = yellow]
      \draw [shade, ball color = \ColorFace ] (0,0) circle (\RadiusOutline);
    \end{scope}

    %=============[绘制嘴巴]==============
    \begin{scope}
      [very thin, yellow]
      \draw (\RadiusMouth,0)
            arc [start angle = 0, end angle = -180, radius = \RadiusMouth];
    \end{scope}

    %=============[绘制眼眶和填充]==============
    \begin{scope}
      [very thin, draw = black!10!white, ball color = white]
      \node (NodeOrbitLeft) at \CoordinateOrbitLeft {};
      \node (NodeOrbitRight) at \CoordinateOrbitRight{};
      %
      \node (Node Nail Left) at ($(NodeOrbitLeft) + (0,\RadiusOrbitOut)$){};
      \node (Node Nail Right) at ($(NodeOrbitRight) + (0,\RadiusOrbitOut)$){};

      % - - - - - - [左眼填充]- - - - - - -
      \begin{scope}
        \OrbitClip{Node Nail Left}
        \shade  (0,0) circle (\RadiusOutline);
      \end{scope}

      % - - - - - - [右眼填充]- - - - - - -
      \begin{scope}
        \OrbitClip{Node Nail Right}
        \shade  (0,0) circle (\RadiusOutline);
      \end{scope}

      % - - - - - - [绘制眼眶框线]- - - - - - -
      \OrbitDraw{Node Nail Left}
      \OrbitDraw{Node Nail Right}

    \end{scope}

    %=============[绘制眼珠]==============
    \begin{scope}
      [very thin, draw = red!10!black, fill = red!30!black]
      \node (NodePupilLeft) at \CoordinatePupilLeft {};
      \node (NodePupilRight) at \CoordinatePupilRight {};
      \filldraw (NodePupilLeft) circle ({\RadiusPupil});
      \filldraw (NodePupilRight) circle ({\RadiusPupil});
    \end{scope}

    %=============[绘制眉毛]==============
    \begin{scope}
      [fill = black]
      \node (Node Start Right Brow) at (\XStartBrow , \YStartBrow)   {};
      \node (Node End Right Brow)   at (\XEndpBrow  , \YEndBrow)     {};
      \node (Node Top Right Brow)   at (\XTopBrow   , \YTopBrow)     {};
      \node (Node Below Right Brow) at (\XTopBrow   , \YTopBrow-\ThickBrow){};

      \node (Node Start Left Brow) at (-\XStartBrow, \YStartBrow)  {};
      \node (Node End Left Brow)   at (-\XEndpBrow , \YEndBrow)    {};
      \node (Node Top Left Brow)   at (-\XTopBrow  , \YTopBrow)    {};
      \node (Node Below Left Brow) at (-\XTopBrow  , \YTopBrow-\ThickBrow){};

      \fill (Node Start Right Brow)
            parabola bend (Node Top Right Brow) (Node End Right Brow)
            parabola bend (Node Below Right Brow)(Node Start Right Brow)
            -- cycle;
      \fill (Node Start Left Brow)
            parabola bend (Node Top Left Brow) (Node End Left Brow)
            parabola bend (Node Below Left Brow)(Node Start Left Brow)
            -- cycle;
    \end{scope}

    %=============[绘制红晕]==============
    \begin{scope}
      [ opacity = 0.3, inner color=red,outer color = \ColorFace]
      \node (Node Right Blush) at \CoordinateRightBlushCcentre {};
      \node (Node Left Blush) at \CoordinateLeftBlushCcentre {};

      \shade (Node Right Blush)
            ellipse [x radius = \XRadBlush, y radius = \YRadBlush];
      \shade (Node Left Blush)
            ellipse [x radius = \XRadBlush, y radius = \YRadBlush];
    \end{scope}

  \end{scope}
}
